\documentstyle[a4,11pt,alltt,fancybox]{report}
%\usepackage{parskip}
\title{The Revised Middl Manual}
\author{Timothy Roscoe and Dickon Reed}
\date{1st Edition - April 1999}

\input psfig
\def\MIDDL{{\sc Middl}}

\newcommand{\idltabbing}[1] {
\begin{tabbing}
\=xxxxx \=xxxxx \=xxxxxxxxxxxxx \=xxxx \=xxxxxxxxxxxxx \=x
\kill #1 \\
\end{tabbing} }

\newcommand{\idlfragment}[1] {{\tt \idltabbing{#1}}}
\newcommand{\idlliteral}[1] {{\em $<$#1$>$}}
\newcommand{\idloptliteral}[1] {{\em [$<$#1$>$]}}
\newcommand{\idlfield}[3]{\> \> \> #1 \> #2 \> #3 }
\newcommand{\Middlc}{{\tt Middlc1} }

\begin{document}
\maketitle
\begin{abstract}

The syntax, type system and facilities of the \MIDDL{} interface
definition language (based loosely on ANSAware IDL) are described.
A set of tools for processing the language has been produced and are
described. A short tutorial on the language is included. \MIDDL{} is also
compared with OMG IDL as used by CORBA.

\end{abstract}

%%%%%%%%%%%%%%%%%%%%%%%%%%%%%%%%%%%%%%%%%%%%%%%%%%%%
\chapter{Introduction}
%%%%%%%%%%%%%%%%%%%%%%%%%%%%%%%%%%%%%%%%%%%%%%%%%%%%

 This document describes the \MIDDL{} interface definition language used
in the Nemesis operating system being developed as part of the Pegasus
project and related projects at the University of Cambridge.

\MIDDL{} started as an attempt to add sufficient functionality to ANSA
IDL for it to be useful in the Nemesis operating system and also the
MSRPC2 remote procedure call system. Unlike most
IDLs it is not based on C or C++, and avoids
implementation issues (the only implementation directive was inherited
from ANSA IDL and has now been removed). The language does not support
constant declarations, but 
power sets of enumerations have been added as an abstraction of
bit-masks.  

\section{Differences from OMG IDL}

The CORBA specification specifies an IDL (the OMG IDL) which may be
familiar to a number of readers. The key differences are:

\begin{itemize}
\item The syntax is derived more from the ALGOL language family than
from C.
\item \MIDDL{} is not preprocessed. 
\item \MIDDL{} has a much more tightly defined mechanism for relating one
interface to another; \MIDDL{} interfaces may either {\tt EXTEND} or {\tt
NEED} other interfaces. 
\item All \MIDDL{} interface names exist notionally in the same namespace. There is no notion of a module in \MIDDL{}.
\item A keyword ({\tt IREF}) is necessary to turn an interface name in
to a type in \MIDDL{}. In OMG IDL interface names may be used
directly as type names.
\item All \MIDDL{} interfaces are required to be (i.e. should be) documented.
\end{itemize}

%\section{Differences from ANSA IDL}
%
%The IDL itself was originally based on the ANSA IDL from ANSAware 4.1.
%The following modifications have been made to date:
%\begin{itemize}
%\item The syntax for invocations and subtyping have been shortened.
%\item Concrete data type for machine addresses and words have been added.
%\item The power set and reference type constructors have been included.
%\item Exceptions are supported for interrogation operations.
%\item Arguments and results can be declared mutable or immutable.
%\item The ANSA inheritance syntax and {\tt FROM} keyword are not allowed.
%\item Interface references are syntactically type specifiers and not
%  type constructors.
%\item Multiple supertypes are not allowed.
%\item Nemesis module specfications are supported.
%\item Type names are scoped.
%\end{itemize}

%\section{Differences from June 1993 \MIDDL{}}
%
%\MIDDL{} has changed somewhat since the first \MIDDL{} manual (June 1993).
%The important changes are:
%\begin{itemize}
%\item Different reserved words for subtyping and operations.
%\item Concrete data type for machine addresses and words have been added.
%\item Exceptions are now included from {\tt NEED}ed interfaces.
%\item The ANSA inheritance syntax and {\tt FROM} keyword are no longer
%  accepted.
%\item Interface references are syntactically type specifiers and not
%  type constructors.
%\item Results of interrogations can now be mutable or immutable.
%\item There is a reference type constructor.
%\item Nemesis module specfications are supported.
%\item Type names are scoped.
%\item The compiler has been completely rewritten, consequently most
%  of the command line options have changed.
%\end{itemize}

%\section{Differences from January 1994 \MIDDL{}}
%
%The important changes are:
%\begin{itemize}
%\item A concrete data type for natural machine words.
%\item Identifiers for exceptions and types are now scoped.
%\item There is a reference type constructor.
%\item Module parameters are specified with a different syntax. In
%  particular, {\tt PARAMETERS} is no longer a reserved word.
%\item The {\tt LOCAL} keyword identifies modules which can use unsafe
%  constructions.
%\item The compiler has (again) been completely rewritten, consequently most
%  of the command line options have changed.
%\end{itemize}

\chapter{Language Tutorial}

Here is an example \MIDDL{} interface, with line numbers. (The line
numbers are not part of the interface).

\begin{verbatim}

01 -----------------------------------------------------------------------------
02 --                                                                          -
03 --  Copyright 1996, University of Cambridge Computer Laboratory             -
04 --                                                                          -
05 --  All Rights Reserved.					            -
06 --                                                                          -
07 -----------------------------------------------------------------------------
08 --
09 -- INTERFACE:
10 --
11 --      Directory.if
12 -- 
13 -- DESCRIPTION:
14 --
15 --      Interface for filesystem directories
16 -- 
17 -- ID : $Id: middl-manual.tex 1.2 Tue, 13 Apr 1999 17:01:06 +0100 dr10009 $
18 -- 
19 -- \chapter{Files}
20 --
21 
22 -- A Directory is an extension of the Context interface which maps
23 -- UNIX-like filenames to File objects. It differs from Context in
24 -- that it provides the additional methods to open/create entries in
25 -- a Directory with options. 
26 
27 Directory : LOCAL INTERFACE =
28  EXTENDS Context;
29  NEEDS File;
30 BEGIN
31	-- "List" does what you would expect: returns a list of
32	-- all of the names in this directory.
33
34	-- "Get" looks up a name in the directory to obtain an
35	-- "File.Attributes" record.  i.e. the equivalent of "stat"
36
37	-- Open options which may be specified
38	OpenOption: TYPE = {ReadOnly, WriteOnly, ReadWrite, Create, Append};
39	OpenOptions: TYPE = SET OF OpenOption;
40
41	Open: PROC [ name: STRING,
42                   options: OpenOptions]
43	      RETURNS [ file: IREF File ]
44            RAISES Context.NotFound, Context.Exists, Context.Denied;
45
46      -- "Open" looks up "name" in the directory and, if it is a 
47      -- regular file, creates a "File" object which may be 
48	-- used to access the file.
49
50	OpenDir: PROC [ name: STRING ]
51	      RETURNS [ file: IREF Directory ]
52            RAISES Context.NotFound, Context.Denied;
53
54      -- "OpenDir" looks up "name" in the directory and, if it is another 
55      -- directory, creates a new "Directory" object.
56
57
58	SimpleOpen: PROC [ name : STRING ]
59	      RETURNS [ file: IREF File ]
60            RAISES Context.NotFound, Context.Exists, Context.Denied;
61
62	-- Same as "Open", but without options. Defaults to "ReadOnly".
63
64 END.
\end{verbatim}

Lines 1 to 20 are merely comments. Comments in \MIDDL{} begin with two
dashes are proceed to the end of the line. We recommend that
interfaces begin with any copyright notices, followed by the interface
name and a brief one line description. This makes browsing interfaces
easier.

Line 21 is blank; blank lines are permitted at any stage and have no
meaning to the language. However, the \MIDDL{} documentation system
ignores all comments up to a blank line.

Lines 22 to 25 are again comments. By convention, we describe the
functionality of the interface here. Some interfaces may have long
multiple paragraph descriptions, with comment lines with no text used
to separate them.

All interfaces begin with a line like line 27. {\tt Directory} is the
name of the interface. The keyword {\tt LOCAL} is optional. It enables
certain language features (most noticeably reference types, addresses and machine words) which cannot be marshaled between
machines. 

Lines 28 and 29 define how the {\tt Directory} interface relates to
other interfaces. The {\tt NEEDS} keyword means that types and
exceptions of the {\tt File} interface may be used in this
interface. The use of {\tt IREF File} in lines 43 and 59 would not be
possible unless the {\tt File} interface had been mentioned in these
lines. 

The {\tt EXTENDS} keyword means that this interface subtypes
another. In this case, line 28 means that all operations of the {\tt
Context} interface will be implemented by all objects exporting a {\tt
Directory} interface. It also has the same effect as writing {\tt
NEEDS Context;}. That is, all the types and exceptions of {\tt
Context} may be used in this interface.

The interface itself begins at line 30, with the {\tt BEGIN}
keyword. A \MIDDL{} interface consists of a number of declarations. On
line 38, we find a declaration of a new type called {\tt
OpenOption}. It is an enumeration; a value of type {\tt OpenOption}
may be one of {\tt ReadOnly, WriteOnly, ReadWrite, Create} or {\tt
Append}. Line 39 declares a second type; a set of {\tt OpenOption}s. A
value of type {\tt OpenOptions} is any subset of all the enumeration
values of {\tt OpenOption}.

On line 41, we find an operation declaration. {\tt Open} will be the
first new operation of {\tt Directory}. Remember that this interface
extends {\tt Context} so any operations of {\tt Context} will also be
operations of {\tt Directory} anyway. {\tt Open} is a procedure; it
takes a string and a {\tt OpenOptions} as arguments and if successful
returns an {\tt IREF} to a file interface. An {\tt IREF} is a
reference to an invocation point which is a (i.e. has type of) {\tt
File}. In practice, this means that the caller is given back a pointer to a
closure which the caller may use to invoke the operations of the {\tt
File} interface. Finally, line 44 specifies what happens when the
operation fails; three exceptions may be raised; {\tt
Context.NotFound, Context.Exists} and {\tt Context.Denied}. In the
{\tt Context} interface, the {\tt NotFound} exception was declared
like this:

\begin{verbatim}
   NotFound   : EXCEPTION [ name : REF CHAR ];
\end{verbatim}

So, if {\tt Open} fails with a {\tt NotFound} exception, the caller
will be able to access a pointer to a character. As with a lot of
things about \MIDDL{}, how this can be done is a property of the language
mapping used. In fact, this is the whole point of \MIDDL{}. By separating
it from C, for instance, developers are less likely to suffer
preconceived notions of how things like exceptions work. Other
documents describe the mapping from \MIDDL{} to C, for instance, or to
Clanger.

The best way to learn \MIDDL{} is to inspect the corpus of interfaces
written in \MIDDL{}. The Nemesis distribution ships with more than 200
of them which together use practically all language facilities of
Nemesis. Refer to the Nemesis interface manual (\cite{XXX}) for
further details. The rest of this manual will describe \MIDDL{} in
detail for reference purposes.

%%%%%%%%%%%%%%%%%%%%%%%%%%%%%%%%%%%%%%%%%%%%%%%%%%%
\chapter{Concrete Types}
%%%%%%%%%%%%%%%%%%%%%%%%%%%%%%%%%%%%%%%%%%%%%%%%%%%

This chapter describes concrete types in \MIDDL{} (as opposed
to the types of interfaces themselves, which are abstract types).
Concrete types are those whose representation is reasonably explicit.
They do include interface references (see below).

% % % % % % % % % % % % % % % % % % % % % % % % % % 
\section{Primitive Types}
% % % % % % % % % % % % % % % % % % % % % % % % % % 

%  \MIDDL{} provides the same 11 primitive types as ANSA IDL. The only
%real differences are that {\tt LONG INTEGER} and {\tt LONG CARDINAL}
%types really are 64 bits in \MIDDL{}, and  there are types for
%machine addresses and words. 

The types are:

\begin{tabbing}
\= xxxxxxxxxxxxxxxxxxxxxxxxxx \= x \kill \\
\> {\tt BOOLEAN}: \> A boolean quantity with values {\tt TRUE} or {\tt
FALSE}. \\
\> {\tt SHORT CARDINAL}: \> A 16-bit unsigned integer. \\
\> {\tt CARDINAL}: \> A 32-bit unsigned integer.\\
\> {\tt LONG CARDINAL}: \> A 64-bit unsigned integer.\\
\> {\tt SHORT INTEGER}: \> A 16-bit signed integer.\\
\> {\tt INTEGER}: \> A 32-bit signed integer.\\
\> {\tt LONG INTEGER}: \> A 64-bit signed integer.\\
\> {\tt REAL}: \> A 32-bit IEEE single precision floating point number.\\
\> {\tt LONG REAL}: \> A 64-bit IEEE double precision floating point number.\\
\> {\tt OCTET}: \> An 8-bit quantity.\\
\> {\tt CHAR}: \> A 7-bit ASCII character.\\
\> {\tt STRING}: \> A variable-length character string. \\
\> {\tt DANGEROUS ADDRESS}: \> A machine address. \\
\> {\tt DANGEROUS WORD}: \> A natural machine word.
\end{tabbing}

Use of the data type {\tt DANGEROUS ADDRESS} is strongly discouraged
unless absolutely necessary. Compilers may refuse to generate
marshalling code for this type. Similarly, {\tt DANGEROUS WORD} will
have different representations on different architectures, and should
be used as little as possible. Both types may only be used in
interfaces declared {\tt LOCAL}. Note that the {\tt DANGEROUS} keyword
is optional.

% % % % % % % % % % % % % % % % % % % % % % % % % % 
\section{Type Constructors}
% % % % % % % % % % % % % % % % % % % % % % % % % % 

New types are defined by a declaration of the form:

\idltabbing{
\idlliteral{new type}
{\tt : TYPE =}
\idlliteral{type constructor} }

Thereafter, the new type can be refered to just as \idlliteral{new
  type} within its defining interface, or as
\idlliteral{intf}.\idlliteral{new type} in some other interface which
{\tt EXTEND}s or {\tt NEED}s the one in which the type is defined.

It is also possible to declare anonymous constructed types as
arguments or results in invocations or components of other types. The 8 type
constructors are as follows; note that the comma is used as a
component separator, not as a terminator:

\subsection{Alias}

Defines a new type as the same as an existing type. The syntax of the
constructor is just the name of the base type. For example:
\idlfragment{Wombat : TYPE = Bat;}

\subsection{Enumeration}

An enumeration is a set of symbols usually mapped onto integer values
in a representation-dependent manner. For example:
\idlfragment{Beer : TYPE = \{ \\
\> \> \> IPA, \\
\> \> \> Abbott, \\
\> \> \> Grolsch \}; }

\subsection{Arrays}

An array is a fixed-size sequence of elements of some base type. For
example:
\idlfragment{MSNLSap : TYPE = ARRAY 8 OF OCTET;}

\subsection{Sequence}

A sequence is a variable-sized sequence of elements of some base type.
For example:
\idlfragment{MachNames : TYPE = SEQUENCE OF STRING;}

\subsection{Record}

A record is a grouping of elements of different base types. For
example: 
\idlfragment {MSRPCInterfaceRef : TYPE = RECORD [ \\
\idlfield{id}{:}{MSRPCInterfaceID,} \\
\idlfield{attrs}{:}{MSRPCAttributes,}\\
\idlfield{hint}{:}{MSRPCAddressHint}]; }

\subsection{Choice}

A choice is a discriminated union. The discriminator must be an
enumeration type. For example:

\idlfragment{MSRPCProtocol : TYPE = \{ \\
\> \> \> MSNL, \\
\> \> \> TCP, \\
\> \> \> SDOM \}; \\
\\
MSRPCAddressHint : TYPE = CHOICE MSRPCProtocol OF \{ \\
\idlfield{MSNL}{=>}{MSNLAddressHint,}\\
\idlfield{TCP}{=>}{TCPAddressHint,}\\
\idlfield{SDOM}{=>}{SDomAddressHint}\}; }

\subsection{Power Set}

A power set is a type whose values are subsets of an enumeration type.
For example:
\idlfragment{BarSelection : TYPE = SET OF Beer;}

\subsection{References}

The {\tt REF} construct defines a type whose values are pointers to
instances of some base type. For example:
\idlfragment{BoatHouse : TYPE = REF BarSelection;}

References may only be used in interface types declared to be {\tt
  LOCAL}.


% % % % % % % % % % % % % % % % % % % % % % % % % % 
\section{Interface References}
% % % % % % % % % % % % % % % % % % % % % % % % % % 

The type expression:
\idlfragment{IREF BeerInterface}
always stands for the unique data type whose values are references to
an interface of type {\tt BeerInterface}. Data types of this kind are
never constructed explicitly, but are created implicitly when a new
interface type specification is parsed. Thus the \MIDDL{} fragment:
\idlfragment{BeerRef : TYPE = IREF BeerInterface}
actually defines an {\em alias} type to the already defined interface
reference type. Note that this discussion is about the {\tt IREF}
type, not values of an interface reference type.

%%%%%%%%%%%%%%%%%%%%%%%%%%%%%%%%%%%%%%%%%%%%%%%%%%%%
\chapter{\MIDDL{} Interface Syntax}
%%%%%%%%%%%%%%%%%%%%%%%%%%%%%%%%%%%%%%%%%%%%%%%%%%%%

  An interface definition file in \MIDDL{} has  this structure:


\begin{tabbing}
\= xxxxxxx \=xxxxxxxxxxxxxxxxx \kill \\
\> \idlliteral{interface name} {\tt :} \idloptliteral{\tt LOCAL} {\tt INTERFACE =} \\
\> \> \idloptliteral{supertype} \\
\> \> \idloptliteral{type imports} \\
\> {\tt BEGIN} \\
\> \> \idloptliteral{type definitions} \\
\> \> \idloptliteral{operation signatures} \\
\>{\tt END.} \\
\end{tabbing}

The {\tt LOCAL} keyword asserts that interfaces of this type will
never be exported outside an address space. This enables use of the
type constructor {\tt REF}, as well as the two concrete types {\tt
  DANGEROUS ADDRESS} and {\tt DANGEROUS WORD}. It is a compile time
error to use these in any imported interface if the top-level
interface has not been declared to be {\tt LOCAL}.

% % % % % % % % % % % % % % % % % % % % % % % % % % 
\section{Supertype Specification}
% % % % % % % % % % % % % % % % % % % % % % % % % % 

A supertype specification looks like:

\idltabbing{
{\tt EXTENDS}
\idlliteral{base interface}
{\tt ;} }

It asserts that this interface type is a subtype of the given base
interface. That is to say, all composite type definitions, attribute
declarations and exception and operation signatures in the base
interface are automatically included in this interface; only
additional types, attributes and operations are defined in this
specification. 

% % % % % % % % % % % % % % % % % % % % % % % % % % 
\section{Type Imports}
% % % % % % % % % % % % % % % % % % % % % % % % % % 

A line in a \MIDDL{} file of the form:

\idltabbing{ {\tt NEEDS}
\idlliteral{other interface}
{\tt ;} }

declares that all type definitions and exception declarations in
the other interface are to be 
made available in this interface, including interface references. Note
that in the previous version of \MIDDL{}, exceptions were not included
from {\tt NEED}ed interfaces.

% % % % % % % % % % % % % % % % % % % % % % % % % % 
\chapter{Operation Declarations}
% % % % % % % % % % % % % % % % % % % % % % % % % % 

There are two kinds of operations supported in \MIDDL{}:
Procedures and Announcements.

\section{Procedures}

Procedures are conventional remote procedure calls. They take
a bunch of arguments, return a bunch of results, and might raise one of
a bunch of exceptions. The general syntax is:
\idltabbing{
\idlliteral{operation}
{\tt : PROC [}
\idlliteral{arguments}
{\tt ]} \\
\> \> {\tt RETURNS [}
\idlliteral{results}
{\tt ]} \\
\> \> [
{\tt RAISES }
\idlliteral{exceptions}
]
{\tt ;} }

  The arguments and results are given as semicolon-separated lists of
declarations. Each declaration is a comma-separated list of
identifiers, followed by a colon and the type specifier. The
declaration can optionally be preceded by one of the keywords {\tt IN}
or {\tt OUT} or both
({\tt IN OUT}).
(see below).

The exceptions are simply a comma-separated list of scoped exception
identifiers. 
Each exception must be defined in the interface (in which case a
simple identifier is used), or in an interface that this one {\tt
  EXTENDS} or {\tt NEEDS}. In the latter cases the name must be
qualified with the interface defining the exception, i.e.
\idlliteral{exc intf}.\idlliteral{exc. name}.

  Here is an example of a fairly full-featured procedure
signature: 

\idlfragment{
\> BuyRound : PROC [ \\
\idlfield{Beers}{:}{BarSelection;}\\
\idlfield{Cider, Lager}{:}{CARDINAL;}\\
\idlfield{MUTABLE Cash}{:}{REAL ]}\\
\> \> RETURNS [ \\
\idlfield{PintsBought}{:}{CARDINAL,}\\
\idlfield{NameOfBarman}{:}{STRING}] \\
\> \> RAISES TimeCalled, Brewery.BeerOff ; 
}

In some cases it may be the case that procedures are never expected to
return, so specifying the return type of he procedure is
meaningless. In such cases, this syntax may be used:

\idltabbing{
\idlliteral{operation}
{\tt : PROC [}
\idlliteral{arguments}
{\tt ]} \\
\> \> {\tt NEVER RETURNS}
{\tt RAISES }
\idlliteral{exceptions}
]
{\tt ;} }

\subsection{Announcements}

  Announcements are a bit like messages; you generate an announcement
and then never hear from it again regardless 
An announcement is declared as:
\idltabbing{
\idlliteral{operation}
{\tt : ANNOUNCEMENT [}
\idlliteral{arguments}
{\tt ] ;} }

  Note that an announcement returns no results and can raise no
explicit exceptions. An example of an
announcement might be:
\idlfragment{
\> CallTime : ANNOUNCEMENT [ \\
\idlfield{Time}{:}{AMOrPM}; \\
\idlfield{LockIn}{:}{BOOLEAN}]; 
}


% % % % % % % % % % % % % % % % % % % % % % % % % % 
\section{Exception Declarations}
% % % % % % % % % % % % % % % % % % % % % % % % % % 

  Exceptions can be raised by procedures as described above.
Exceptions have arguments (to indicate the nature of what went wrong)
and are declared in a similar way to announcements:
\idltabbing{
\idlliteral{operation}
{\tt : EXCEPTION [}
\idlliteral{arguments}
{\tt ] ;} }

  An exception must be declared before any procedure which can
raise it. By their nature they cannot return any results, nor can
they raise exceptions themselves. In addition to the exception list in
an procedure declaration, here may be implicit exceptions which
can be raised by any operation. For example, any operation in MSRPC2
({\em including} announcements) can raise the following exception:
\idlfragment{
\> InternalError : EXCEPTION [ \\
\idlfield{Status}{:}{CARDINAL}; \\
\idlfield{err\_no}{:}{INTEGER}]; 
}


% % % % % % % % % % % % % % % % % % % % % % % % % % 
\section{Argument passing modes}
% % % % % % % % % % % % % % % % % % % % % % % % % % 


Arguments to {\tt PROC}s can optionally be declared as {\tt IN, OUT,}
or {\tt IN OUT}. The exact meaning of these words is defined by the
programming language mapping but generally:

\begin{itemize}
\item {\tt IN} declares an argument to be passed in by value. The
server takes a copy of the argument.

\item {\tt OUT} declares an argument to be assigned to by the
server. This is different from operation results, since the storage
for an {\tt OUT} argument has been allocated in advance by the
client. Storage for the results must be allocated by the server.

\item {\tt IN OUT} arguments should be considered as being passed by
value-return: the value is copied into the server, modified, then
written back. This may be implemented by call-by-reference.
\end{itemize}

Note that {\tt IN} is the default.


%%%%%%%%%%%%%%%%%%%%%%%%%%%%%%%%%%%%%%%%%%%%%%%%%%%%
\appendix
\chapter{Lexical Symbols}
%%%%%%%%%%%%%%%%%%%%%%%%%%%%%%%%%%%%%%%%%%%%%%%%%%%%

The following strings are reserved words in \MIDDL{}:
\begin{verbatim}

ADDRESS ANNOUNCEMENT ARRAY BEGIN BOOLEAN CARDINAL CHAR CHOICE
DANGEROUS END EXCEPTION EXPORTS EXTENDS IMMUTABLE IMPORTS INTEGER
INTERFACE IREF LOCAL LONG MODULE MUTABLE NEEDS NEVER OCTET OF PROC RAISES
REAL RECORD REF RETURNS SEQUENCE SET SHORT STRING TYPE

\end{verbatim}

The following punctuation symbols have a special meaning:
\begin{verbatim}
:  =  ;  [  ]  .  {  }  ,  =>  --
\end{verbatim}

Tokens are separated by whitespace, which consists of spaces, tabs,
newlines and comments. Comments start with ``{\tt --}'' and end
with a newline.

Numbers are only composed of the digits 0-9.

Identifiers are sequences of alphanumeric characters or ``{\tt \_}''.
They may not start with a digit.


%%%%%%%%%%%%%%%%%%%%%%%%%%%%%%%%%%%%%%%%%%%%%%%%%%%%
\chapter{BNF Grammar}
%%%%%%%%%%%%%%%%%%%%%%%%%%%%%%%%%%%%%%%%%%%%%%%%%%%%

\VerbatimInput{NewGrammar}

%%%%%%%%%%%%%%%%%%%%%%%%%%%%%%%%%%%%%%%%%%%%%%%%%%%%%%%%%%%%%%%%%%%%%%
\chapter{\MIDDL{} Tools and Resources}
%%%%%%%%%%%%%%%%%%%%%%%%%%%%%%%%%%%%%%%%%%%%%%%%%%%%%%%%%%%%%%%%%%%%%%
\begin{description}

\item[cl/astgen] is the compiler used to build a number of products
including C header files, Nemesis typesystem records, stubs and
skeleton closure implementations from \MIDDL{} files. It was written by
Dickon Reed initially as an undergraduate project and developed
further by Richard Mortier. It also includes partial support for a
very experimental systems programming language called CLUless. 

The front end, cl, parses \MIDDL{} interfaces and generates {\tt .ast}
files which contain a bytecode form of the interface. The back end,
astgen, reads multiple {\tt .ast} files and output specifications and
executes the output specifications to generate products. astgen is
designed to make writing new output specifications
straightforward. Output specifications are python classes, in files
prefixed with {\tt gen} and postfixed with {\tt .py}. 

The Nemesis build system includes support for invoking cl and astgen
efficiently. It is sometimes useful to invoke these tools manually. To
parse a file, try something like:

\begin{verbatim}
cl Foo.if
\end{verbatim}

To generate a skeleton C file that implements a closure of type
{\tt Foo}, execute:

\begin{verbatim}
cl -A Foo.if
astgen c Foo.ast
\end{verbatim}

\item[\MIDDL{}c] is the compiler written by Timothy Roscoe. It is similar
in many ways to cl/astgen, but is generally simpler and is more self
contained. It is kept around mostly for reference purposes.

\item[ifrefmanualgen.py] is a script to convert a number of interfaces
to a \LaTeX file. It recursively finds all interfaces (i.e. files
ending in {\tt .if}) in the current directory and below and produces a
single LaTeX file on stdout.

\item[The Nemesis Interface Manual] is the output of {\tt ifrefmanualgen}
on all Nemesis interfaces. It is a useful source of examples of \MIDDL{}.

\item[The Nemesis C programmers Manual] describes the \MIDDL{} to C mapping.

\end{description}

\end{document}

