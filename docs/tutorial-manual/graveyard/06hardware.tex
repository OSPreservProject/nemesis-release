% appendix to nemesis intro doc

\chapter{Supported Hardware}
\label{sec:hardware}

All of the information provided in this section is to the best of our
knowledge and belief.  We do not guarantee that it is error-less nor
are we responsible for problems derived from purchasing according to
this information.

\section{Systems}

The support for various platforms is summarised in table
\ref{tab:support}.  \emph{There are some 'issues' with PCI and Intel
  hardware.}  In particular the Triton I and Triton II chip-sets
contain some PCI bugs which prevent certain PCI cards from working
correctly.  In particular it is known that the Digital ATM Works 350
does not work with some: see table \ref{tab:workingpci} for details.  The
Triton is the name for the Pentium's motherboard chip-set, the Natoma
for the Pentium Pro.  We therefore recommend Natoma based machines.
  
\begin{table}[htb]
\begin{center}
\begin{tabular}{c|c}
Platform & Support \\ \hline
Intel & Full \\
Digital PC164 & Full \\
Digital EB164 & Full \\
Digital EB64 & Little \\
Digital AXP3000 & Moderate \\
Digital SHARK & Moderate \\
Digital IT & Moderate \\
Intel Multiprocessors & Little \\
\end{tabular}
\end{center}
\caption{Platforms listed in order of support}
\label{tab:support}
\end{table}

\begin{table}[htb]
\begin{center}
\begin{tabular}{c|c|c}
Working    & Broken     & Unknown \\ \hline
82441FX    & 82437      & 430TX       \\
AL440LX    & 82439FX    & 430VX       \\
           & 82439HX    & 82789IB     \\
\end{tabular}
\end{center}
\caption{Summary of PCI working-ness}
\label{tab:workingpci}
\end{table}

\emph{If you have doubt about your machine boot linux and run
  \texttt{grep 'Host bridge' /proc/pci}, the output should contain
  exactly one of the words ``Triton'' or ``Natoma''.}

There are a number of comments:
\begin{itemize}
\item The extant alpha pal-code is for EV4, EV45, EV5 and EV56 chips.
\item We do not recommend purchasing Digital hardware from
  ``Bytech'', rather we currently recommend ParSys Ltd.
\item The Intel port has not been tested recently on 386 or 486 hardware.
\end{itemize}

\section{Recomendations}

We recommend either the purchase of Intel Natoma (Pentium Pro or
P{\small II}) based
systems,  or the purchase of Digital Alpha PC164 systems.

\section{Device Hardware}

Existing device drivers for various hardware on various busses are
described below.  Partners are encouraged to write additional device
drivers.

\subsection{PCI}
\begin{itemize}
\item The most recent frame buffer supported is the Matrox
  Millennium~II.  The Diamond Stealth 64 2Mbyte VRAM card based on the
  S3/968 chip-set using ``New MMIO'' with either the IBM RGB524 timing
  chip, or the TI 3025/6 is also supported.
\item The Ethernet cards supported are those based on the DE4x5
  chip-sets.  In particular the DE450 and DE500 (the latter supporting
  both 10 and 100Mbit) have been tested in both EB164 and
  Intel\footnote{At the time of writing the bootp client has yet to
    have support for this card integrated.}.
\item The OPPO ATM card (Digital ATM Works 350) is supported in all
  PCI machines except the Triton.  This card is highly recommended.
  On \emph{no} account get the Digital ATM Works 350{\bf L} card, which is
  completely different and neither supported nor recommended.
\item The Nicstar ATM card (based on the IDT 77201) is supported, but
  has much lower performance than the OPPO.  Ensure you get revision E
  (or later) silicon.
\item Most cards based on the NCR53c8xx SCSI chip are supported.
\end{itemize}

\subsection{ISA}

Note that like similar restrictions with Linux, ISA cards which
require strange BIOS Pnp activities, (e.g. some soundblaster cards),
may not work in machines WITH no BIOS (e.g. EB164s).  In general the
purchase of ISA PnP cards should be avoided at least for the time
being until plug and play support has been implemented under Nemesis.

\begin{itemize}
\item For PCs, the 3c509 Ethernet card is supported.  Before Nemesis
  uses it, it should be configured correctly by its DOS install
  program.
\item The NS16550A UART serial chip is supported.  A serial mouse may
  be used on COM2 on Intel platforms.
\item The Standard ISA British Keyboard is supported.  Windows silly
  keyboards with too little space bar are not fully supported~--- not
  all keysyms work.
\item PS/2 mouse is supported (Intel and EB164).
\item Sound blaster 16 or \emph{hardware} compatibles are supported.
\item An am79c960 ``Lance'' Ethernet driver exists but will only work
  on the EB64.
\item The cs8900 ethernet card for Digital Sharks is supported.
\end{itemize}

\subsection{TURBOchannel and other AXP3000 buses}
\begin{itemize}
\item The (P)OTTO ATM card (Digital ATM Works 750) is supported.  This
  is the only supported ATM card for this bus.
\item The ``lance'' Ethernet chip on the AXP3000 Junk-I/O-bus is
  supported.
\item The UART on the AXP3000 Junk-I/O-bus is supported.
\end{itemize}

\subsection{PCMCIA}

A driver is available for the 3c589B ethernet card.

\subsection{Podule}

A podule bus serial line driver is available.  With the demise of the
RISC\,PC, it is unlikely that there will be many further developments
for this bus.
